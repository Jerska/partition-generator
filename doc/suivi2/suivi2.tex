\documentclass[12pt]{article}
\usepackage[francais]{babel}
\usepackage[T1]{fontenc}
\usepackage[utf8]{inputenc}
\usepackage[top=3cm, bottom=2cm, left=2cm, right=2cm]{geometry}
\usepackage[babel]{csquotes}
\usepackage[normalem]{ulem}
\usepackage{graphicx}
\usepackage{eurosym}
\usepackage{tabularx}
\usepackage{gensymb}
\usepackage{url}
\setlength{\headsep}{0.5in}
\renewcommand{\baselinestretch}{1.35}
\setlength{\footnotesep}{0.4cm}
\setlength{\skip\footins}{1cm}
\MakeAutoQuote{«}{»}
\pagestyle{headings}
\graphicspath{{img/}}

\begin{document}

\pagenumbering{gobble}
\begin{titlepage}

	\flushleft{
		\textit{
			17 Mai 2014
		}
	}

	\flushright {
		GONZALEZ Julien \emph{(gonzal\_j)} \\
		DELANNOY Antoine \emph{(delann\_a)} \\
		DE CAMY Matthieu \emph{(de-cam\_m)} \\
		CAVALIÉ Quentin \emph{(cavali\_q)} \\
		DUMONT Matthieu \emph{(dumont\_h)} \\
		\textbf{EPITA 2015}
	}

	\begin{center}
		\vspace*{\stretch{1}}

		{
			\Huge
				\textsc{
					\textbf{
						Projet de Fin d'Études en Entreprise \\
					}
				}
		}
		{
			\LARGE
				\textsc{
					\textbf{
						Présentation pour la réunion GO/NoGO \\
					}
				}
		}

		{
				\includegraphics[width=100px]{jellynote.png}\hspace{3cm}\includegraphics[height=50px]{epita.jpg}\\
				[1cm]
		}

		\textbf{
			\textit {
				Création d'un outil de génération de partition à partir d'un fichier audio.\\
				[1cm]
			}
		}
		\vspace*{\stretch{1}}
	\end{center}
	\flushright {
		\textbf{Dates du projet:} \textit{de Mai 2014 à Janvier 2015}
	}
\end{titlepage}

\pagenumbering{arabic}

\newpage
\tableofcontents

\newpage
\clearpage

\section{Introduction}

\par Ce rapport a pour but de faire état de l'avancement de notre PFEE, de présenter les obstacles encontrés et ceux à venir, et donc de faire une synthèse de ce qui a été fait et ce qu'il reste à faire.

\par Le dernier suivi s'étant mal déroulé par manque de travail de notre part, nous espérions avoir fait le nécessaire pour que cela ne se reproduise plus. Un obstacle majeur est cependant survenu: nous avons perdu un membre de notre équipe, qui a quitté la majeure SCIA pour faire une année de césure. Nous avons donc du réorganiser notre planning et la répartition des tâches pour pallier à ce départ.

\par La fin de cette période de travail a aussi été marquée par une absence de notre contact à Jellynote Victor Lenoir (probablement parti en vacances), qui nous a contraint à devoir retarder l'avancement de certaines parties du projet qui nécessitent sa mise à contribution.

\section{Problématiques}
\label{sec:prob}

\par Nous avons démarré cette période de travail par une visite à Jellynote pour discuter avec Victor des différentes problématiques pouvant poser problème pour l'avancement du projet. Il en est ressorti plusieurs points, dont voici les plus pertinents:\\

\begin{itemize}
\item Il nous a été donné du code d'implémentation naïve de détection de note. Ce code nous sert de référence et de base pour avancer dans ce projet, bien que nous ne nous servirons à aucun point de ce code dans le projet. En effet, notre projet doit pouvoir détecter de quelle note il s'agit, alors que le leur ne fait que valider si une note est bien celle attendue ou non. Ceci pose des problématiques différentes, et nous ne pouvons que nous inspirer de ce qui a été fait au sein de l'entreprise pour trouver des pistes pertinentes à notre projet.\\
\item L'utilisation de \textbf{Lilypond} pour générer les partitions est une problématique bien plus importante que ce que nous avions envisagé. En effet, la génération de partition chez Jellynote est faite côté serveur et n'a pas de contraintes de temps-réel, ce qui cause des problèmes pour l'adapter à notre utilisation. \\
\par L'absence de Victor sur la fin de la période de travail nous a contraint à retarder la conception technique finale de ce module. Nous avons cependant en tête plusieurs plans d'action:
\begin{enumerate}

\item L'équipe Jellynote s'est déjà penché sur ce problème, et Victor nous avait évoqué l'existence en interne d'un script permettant de mettre à jour les partitions lilypond en temps-réel. Mais, sans nouvelles depuis deux semaines, nous n'avons pas pu compter là-dessus, d'où la recherche d'autres moyens de contourner ce problème (énoncés ci-dessous).

\item Nous avons plusieurs pistes pour réduire le temps de génération des partitions avec Lilypond, mais ces techniques sont compliquées à implémenter et nous permettraient uniquement de réduire le temps de génération à environ 500ms. Si nous n'arrivons pas à avoir de l'aide de Jellynote sur ce point, nous aurons surement recours à cette technique, mais nous n'avons pas voulu nous lancer dessus en attendant d'avoir un retour de Victor.

\item L'utilisation de \textbf{ABCjs} plutôt que Lilypond. Ce programme est similaire à Lilypond, si ce n'est qu'il est lui bien plus adapté à la mise à jour en temps réel. Un des problèmes est qu'ABCjs n'est plus maintenu et manque de stabilité. L'autre problème, et celui qui nous pousse à écarter cette solution, est que Jellynote n'utilise que Lilypond pour sa génération, et qu'il est donc compliqué de ne pas s'appuyer sur ce dernier par soucis de cohérence au sein de l'univers Jellynote. Nous gardons cependant cette possibilité en dernier recours si une solution n'a pas été trouvée d'ici là. \\


\end{enumerate}

\item La partie enregistrement au microphone sera géré en Flash. Dans la version actuelle ceci est fait en JavaScript, mais nous comptons migrer en Flash d'ici le prochain suivi. Ceci permettra d'avoir une meilleure précision au niveau des fréquences captées par le microphone. Nous avons décidé de gérer le microphone directement en utilisant les technologies web car cela est bien plus simple à gérer qu'en faisant du C++.

\end{itemize}

\section{Le travail effectué}

\subsection{Génération du code JavaScript}

\par Notre projet doit fonctionner dans un navigateur web, ce qui fait que notre projet doit être \emph{in fine} compilé en JavaScript. Nous procèderons donc ainsi:
\begin{itemize}

\item La partie enregistrement du son est faite directement en \textbf{Flash} et \textbf{js}.
\item Le traitement du son est faite en \textbf{C++}, qui génère ensuite du js grace à \textbf{emscipten}.
\item La génération de la partition sera faite côté serveur (nous nous réservons la possibilité de changer ce point, comme expliqué en partie \ref{sec:prob}).
\item Le lien entre les modules purement js et les modules js généré à partir du C++ est également géré grâce à emscripten.

\end{itemize}

\par Nous n'avons pas encore lié toutes les parties car cela aurait été trop bloquant pour l'instant, et nous avons donc privilégié le développement par module indépendants, que nous modifierons par la suite pour pouvoir les intégrer au livrable final.

\subsection{Intégration web et microphone}

\par Le premier module ainsi développé est la partie purement js. Il s'occupe de gérer l'accès au microphone par le navigateur web. Actuellement, cette partie a ces fonctionnalités:
\begin{itemize}

\item Enregistrement du son \emph{via} le microphone en utilisant JavaScript.
\item Application et visualisation d'une Fast-Fourier Transform (FFT) sur le signal d'entrée.
\item Algorithme naïf de réduction de bruit.
\item Détection basique de la note et affichage dans la console si une note a été détectée.

\end{itemize}

\par \`A terme, ce module devra fonctionner ainsi:
\begin{itemize}

\item Enregistrement du son \emph{via} le microphone en utilisant Flash à la place de js.
\item Envoi du son sans application de FFT au module de détection de notes.
\item Affichage des partitions générées par le module de génération des partitions.
\item Eventuellement gestion de la liaison avec le serveur si besoin.

\end{itemize}
\subsection{Détection des notes}

\par Ceci est le module le plus important du projet, et celui qui sera le centre de tous nos efforts au fur et à mesure de l'avancement du projet. L'algorithme est pour l'instant assez basique mais permet de détecter des notes simples et uniques, de la manière suivante:

\begin{itemize}

\item Lecture d'un fichier audio au format .wav.
\item Application d'un fenêtrage de Hamming (nécessaire pour améliorer la qualité de la FFT).
\item Application d'une FFT sur la totalité du fichier d'entrée.
\item Transformation des résultats obtenus par la FFT en Spectrogramme.
\item Extraction de la fondamentale \emph{via} calcul du produit spectral harmonique.
\item Conversion de la fondamentale en la note associée.

\end{itemize}

\par Pour l'instant, nous n'avons testé que sur des notes uniques, mais nous arrivons à bien détecter la quasi-totalité de celles-ci.\\

\par Bien que nous ne pouvons pas détailler la manière dont fonctionnera l'algorithme final (puisque cela demande beaucoup de recherche et d'essais), nous savons cependant qu'il nous faudra changer certaines choses pour pouvoir coller au cahier des charges:
\begin{itemize}

\item Il faudra lier ce module avec le microphone. Nous garderons également sans doute la lecture de fichier .wav, plus utile pour les tests automatisés.
\item Afin de faire du temps-réel, nous ne pourrons pas appliquer la FFT sur l'ensemble du signal, et devrons donc appeller cette fonction toutes les \emph{x} ms. Cela entrainera également pas mal de changements sur la détection de la note.
\item Il faudra également trouver un moyen de bien placer la note dans la partition, et donc réussir à garder la trace du temps écoulé.
\item S'ajoute à tout ceci la détection de sons plus compliqués qu'une simple note, ie. accords, gestion du bruit etc.

\par Nous avons d'ores et déjà réuni plusieurs papiers scientifiques qui nous permettront de choisir la méthode la plus appropriée pour ce module. Vous trouverez les références de ces études à la fin de ce rapport.

\end{itemize}

\subsection{Génération des partitions}

\par Comme expliqué précédemment, nous sommes en attente d'une réponse de notre contact à Jellynote pour pouvoir se lancer réellement dans la génération des partitions. Cependant, nous avons tout de même développé un petit script permettant de générer du code Lilypond. Il n'est pas encore lié avec le module de reconnaissance des notes mais démontre notre capacité à générer ce type de code.

\section{Objectif pour la prochaine séance}

\par L'objectif principal pour la prochaine séance sera de lier tous les modules développés entre eux. Nous espérons également pouvoir se poser avec Victor pour faire un point et pouvoir réellement avancer sur la génération de partition. Si tout cela est fait, nous essaierons alors de commencer à sérieusement désigner l'application finale.
\par Paralèllement, nous continuerons à améliorer l'algorithme de reconnaissance des notes, et également de le passer 
\newpage

\begin{thebibliography}{9}

\bibitem{realtime}
  Choi Ako,
  \emph{Real-time fundamental frequency estimation by least-square fitting}.
  \url{http://hub.hku.hk/bitstream/10722/43639/1/31101.pdf}


\bibitem{qstrans}
  Judith C. Brown, Miller S. Puckett,
  \emph{An efficient algorithm for the calculation of a constant Q transform}.
  \url{http://academics.wellesley.edu/Physics/brown/pubs/effalgV92P2698-P2701.pdf}


\bibitem{realtime}
  Adam M. Stark, Mark D. Plumbley,
  \emph{Real-time chord recognition for live performance}.
  \url{https://www.eecs.qmul.ac.uk/~markp/2009/StarkPlumbley09-icmc.pdf}


\bibitem{realtimelm}
  Takuya Fujishima,
  \emph{Real-time chord recognition of musical sound: A system using common lisp music}.
  \url{http://www.music.mcgill.ca/~jason/mumt621/papers5/fujishima_1999.pdf}


\bibitem{realtimebs}
  Adam M. Stark, Matthew E. P. Davies and Mark D. Plumbley,
  \emph{Real-time beat-synchronous analysis of musical audio}.
  \url{https://www.eecs.qmul.ac.uk/~markp/2009/StarkDaviesPlumbley09-dafx.pdf}


\bibitem{realtime}
  Matthias Mauch, Katy Noland, Simon Dixon,
  \emph{Using musical structure to enhance automatic chord transcription}.
  \url{https://www.eecs.qmul.ac.uk/~simond/pub/2009/ISMIR2009-Mauch-PS2-7.pdf}

\end{thebibliography}

\end{document}

